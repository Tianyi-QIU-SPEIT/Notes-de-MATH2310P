\documentclass{article}

\usepackage{graphicx} 
\usepackage[french]{babel}
\usepackage[T1]{fontenc}
\usepackage[utf8]{inputenc}
\usepackage{lmodern}
\usepackage{microtype}
\usepackage{hyperref}
\usepackage{amsmath}
\usepackage{amssymb}
\usepackage{geometry}
\usepackage{fancyhdr}
\usepackage{ctex}
\usepackage{tcolorbox}
\pagenumbering{arabic}
\pagestyle{fancy}
\fancyhead[L]{École d'Ingénieurs Paris-SJTU}
\fancyhead[R]{Corentin邱天意}
\fancyfoot[C]{\thepage}
\renewcommand{\headrulewidth}{1pt}
\renewcommand{\footrulewidth}{1pt}

\makeatletter
\@addtoreset{section}{part}
\def\@part[#1]#2{%
    \ifnum \c@secnumdepth >\m@ne
      \refstepcounter{part}%
      \addcontentsline{toc}{part}{\thepart\hspace{1em}#1}%
    \else
      \addcontentsline{toc}{part}{#1}%
    \fi
    {\parindent \z@ \raggedright
     \interlinepenalty \@M
     \normalfont\raggedright
     \ifnum \c@secnumdepth >\m@ne
       \LARGE\bfseries \partname\nobreakspace\thepart
       \par\nobreak
     \fi
     \huge \bfseries #2%
     \markboth{}{}\par}%
    \nobreak
    \vskip 3ex
    \@afterheading}
\renewcommand\partname{Topic}
\makeatother

\title{\textbf{Sommaire du Cours : MATH2310P} \\ Cours assuré par Sébastien GODILLON}
\author{Rédigé par Corentin邱天意}
\date{Semestre 2024-2025-2}

\begin{document}

\maketitle

\centerline{\includegraphics[scale=0.4]{sjtu}}



\newpage
\tableofcontents

\newpage

\part{Équations différentielles ordinaires}
\section{Cours 21 févr : Généralités}

\begin{tcolorbox}[colback=red!5!white,colframe=red!75!black,title=Définition 1.1]

Une \textbf{Équation différentielle linéaire(EDO)} est une équation de la forme:
\[
\forall t \in I, F(t, X(t), X'(t)... X^{(k)}(t)) = 0 
\] 

Plus spécifiquement sur les notations:


\begin{itemize}
 \item $X$ est une fonction inconnue, d'une seule variable réelle, et à valeurs réelles ou vectorielles($X: \mathbb{R} \rightarrow \mathbb{R}^{n}, n \in \mathbb{N}^{*}$). Elle est supposée $k$-fois dérivable sur $I$/.
 \item $t$ est la variable da la fonction $X$.
 \item $I \subset \mathbb{R}$, c'est l'intervalle de définition de l'équation différentielle.
 \item $F$ est une fonction de plusieurs variables, elle est fixée.
 \item $k \in \mathbb{N}^{*}$, on l'appelle l'ordre de l'EDO.
 
\end{itemize}

\end{tcolorbox}

\begin{tcolorbox}[colback=cyan!5!white,colframe=cyan!75!black,title=Exemple 1.1]

\textbf{Chercher les primitives}

Soit $f$ une fonction réelle qui est continue sur l'intervalle $I \in \mathbb{R}$. 

D'après the théorème fondamental de l'analyse(TFA), on sait que $f$ admet des primitives sur $I$. 

Alors, trouver des primitives de $f$ revient à résoudre l'EDO:
\[
\forall t \in I, X'(t) = f(t)
\]
ici on a $F(t, X(t), X'(t)) = X'(t) - f(t)$, une EDO d'ordre 1.

\end{tcolorbox}

\begin{tcolorbox}[colback=cyan!5!white,colframe=cyan!75!black,title=Exemple 1.2]

\textbf{L'oscillateur harmonique}

Le mouvement d'un oscillateur harmonique est modélisé par l'EDO:  
\[
\forall t \in I, X''(t) + \frac{k}{m}X(t) = 0
\]
Elle est d'ordre 2. En considérant le problème physique on trouve que: $I = \mathbb{R}_{+}$, et que $X(0)$ est une condition initiale à déterminer. $k$ et $m$ désignent respectivement le raideur du ressort et la masse.

La forme générale s'écrit: $F(t, X(t), X'(t), X''(t)) = X''(t) + \frac{k}{m}X(t)$.

\end{tcolorbox}

\begin{tcolorbox}[colback=cyan!5!white,colframe=cyan!75!black,title=Exemple 1.3]

\textbf{Le pendule simple}

Il est modélisé par l'équation:  
\[
\theta ''(t) + \frac{g}{l} \sin{\theta {(t)}} = 0
\]
Où $l$ est la longeur de la corde, $g$ le module de l'accélération gravitationelle, et $\theta$ l'angle aigu entre la corde et la verticale. Attention, elle n'est pas linéaire car la fonction $\sin$ ne l'est pas.

\end{tcolorbox}

\begin{tcolorbox}[colback=cyan!5!white,colframe=cyan!75!black,title=Exemple 1.4]

\textbf{Dynamique d'une population: Lotka-Volterra}

On se place dans le monde où il n'y a que les proies et les prédateurs.

Notons: $X(t)$ la population des proies et $Y(t)$ celle des prédateurs à l'instant $t$.

On a:
\[
\begin{cases} 
X'(t) = X(t)(\alpha - \beta Y(t))\\
Y'(t) = Y(t)(\gamma X(t) - \eta)
\end{cases}
\]

Il y a deux équations donc posons la fonction vectorielle $Z(t) = (X(t), Y(t))$. La forme générale de notre EDO s'écrit: $F(t, X(t), X'(t)) = Z(t)$. Elle est d'ordre 1.

\end{tcolorbox}



\begin{tcolorbox}[colback=gray!5!white,colframe=gray!75!black,title=Rappel 1.1]

\textbf{Équations différentielles linéaires homogènes d'ordre 1}

Une équation de la forme:

\[
(E_{1}) : X'(t) + a(t)X(t) = 0 
\] 

Où $a$ est une fonction fixée et continue.


\end{tcolorbox}

\begin{tcolorbox}[colback=red!5!white,colframe=red!75!black,title=Théorème 1.1]

Les solutions de $(E_{1})$ sont toutes de la forme: $t \mapsto \lambda e^{-A(t)}$. $\lambda$ est une constante quelconque, et $A$ est une primitive de $a$.
\tcblower
On peut prendre n'importe quelle primitive car la différence entre deux primitives est une constante.

\end{tcolorbox}

\paragraph{Preuve}: \space

Posons deux ensembles: $S_{1} = \{ X | X'(t) + a(t)X'(t) = 0\}$ et $ S_{2} = \{ X : t \mapsto \lambda e^{-A(t)} | \lambda \in \mathbb{R}\}$, où $X$ est une fonction. Montrons que les deux ensembles sont égaux par double inclusion.


\begin{itemize}
 \item ($S_{1} \supset S_{2}$)
 
 Soit $\lambda$ un réel, on pose la fonction $X: t \mapsto \lambda e^{-A(t)}$, donc elle est un élément de $S_{2}$. $X$ est composée des fonctions dérivables, et d'après les théorèmes généraux elle est aussi dérivable.
 
 On a: $X'(t) = \lambda (-A'(t))e^{-A(t)} = - \lambda a(t) e^{-A(t)}$.
 
 Donc: $X'(t) + a(t)X(t) = - \lambda a(t) e^{-A(t)} + \lambda a(t) e^{-A(t)} = 0$, c'est-à-dire que $X \in S_{1}$, et que $S_{1} \supset S_{2}$.

 \item ($S_{2} \supset S_{1}$)
 
 Soit $X \in S_{1}$. Montrons que $X \in S_{2}$.
 
 On cherche une constante réelle $\lambda$, telle que $X(t) = \lambda e^{-A(t)}$.
 
 Posons la fonction $f$ qui à $t$ associe $\frac{X(t)}{e^{-A(t)}}$, c'est-à-dire $f(t) = \frac{X(t)}{e^{-A(t)}} = X(t)e^{A(t)}$. On suppose que la fonction $X$ est 1-fois dérivable car elle est solution d'un équation différentielle, et donc notre $f$ est aussi dérivable comme composée des fonctions dérivables.
 
 On a: $f'(t) = (X'(t) + X(t)a(t))e^{A(t)} = 0$ car $X \in S_{1}$. Donc $f$ est constante, on note $\lambda$ sa valeur.
 
 De plus, $X(t)e^{A(t)} = \lambda, X(t) = \lambda e^{-A(t)} \in S_{2}$. On trouve que $S_{2} \supset S_{1}$.

\end{itemize}

Par double inclusion on trouve le résultat énoncé.


\begin{tcolorbox}[colback=gray!5!white,colframe=gray!75!black,title=Rappel 1.2]

\textbf{Équations différentielles linéaires non-homogènes d'ordre 1}

Une équation de la forme:

\[
(E_{2}) : X'(t) + a(t)X(t) = b(t)
\] 

Où $a$ et $b$ sont des fonctions fixées et continues.


\end{tcolorbox}

\begin{tcolorbox}[colback=red!5!white,colframe=red!75!black,title=Théorème 1.2]

Toutes les solutions de $(E_{2})$ sont de la forme: $X = X_{p} + X_{h}$, où $X_{p}$ est une solution particulière, et $X_{h}$ est une solution de l'équation homogène associée à $(E_{2})$.

\tcblower

On appelle ce résultat \textbf{le principe de superposition}.

\end{tcolorbox}

\paragraph{Preuve}: \space





\begin{tcolorbox}[colback=cyan!5!white,colframe=cyan!75!black,title=Exemple 1.5]

Résoudre l'équation différentielle: $f(t) - tf'(t) = 1$ pour $t \in ]- \infty, 0[ $ ou $ ]0, +\infty[$.

\tcblower

Faites attention: $\mathbb{R}^{*} $ n'est pas un intervalle.

\end{tcolorbox}

\paragraph{Solution}: \space

D'après le théorème 1.2 on sait qu'on doit chercher deux solutions: une particulière et une homogène. On va d'abord manipuler l'équation pour qu'elle soit de la forme générale.

\[
f'(t) - \frac{1}{t} f(t) = - \frac{1}{t}
\]

\begin{itemize}
 \item \textbf{Solution homogène}
 
 Cherchons une solution de l'équation homogène associée: $f'(t) - \frac{1}{t} f(t) = 0$.
 
 D'après le théorème 1.1, $f$ est de la forme: $f : t \mapsto \lambda e^{ln|t|} = \lambda |t|$, avec $\lambda$ une constante quelconque. Ici on trouve le logarithme népérien comme primitive de $\frac{1}{t}$.
 

 \item \textbf{Solution particulière}
 
 On remarque que la fonction constante et égale à 1 est une solution particulière.

\end{itemize}

D'après le théorème 1.2, toutes les solutions sont de la forme: $f : t \mapsto f_{p}(t) + f_{h}(t) = 1 + \lambda |t|$, avec $\lambda$ une constante réelle.




\begin{tcolorbox}[colback=gray!5!white,colframe=gray!75!black,title=Rappel 1.3]

\textbf{ÉDLs homogènes d'ordre 2 à coefficients constantes}

Une équation de la forme:

\[
(E_{3}) : X''(t) + aX'(t) + bX(t)= 0
\] 

Où $a$ et $b$ sont des constantes.

\end{tcolorbox}











\newpage
\part{Courbes et Surfaces}

\end{document}